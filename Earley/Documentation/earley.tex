\documentclass[11pt]{article}
\usepackage[T1]{fontenc}
\usepackage{alltt}
\usepackage{moreverb}
\usepackage{epsfig}
\usepackage{makeidx}
\usepackage{changebar}

\setlength{\parskip}{0.3cm}
\setlength{\parindent}{0cm}

\def\inputfig#1{\input #1}
\def\inputtex#1{\input #1}

\input spec-macros.tex
\def\ReadOnly{This function returns objects that reveal some internal state;
do not modify those objects.\ }

\def\ra{\rightarrow}

\makeindex
\begin{document}

\section{Earley Parser}

\subsection{Introduction}

An Earley parser is a parser that can handle all context-free
grammars, even ambiguous ones.  The worst-case complexity is $O(n^3)$
where $n$ is the length of the input string.  However, this worst-case
behavior is rare, and for a large class of reasonable grammars the
worst-case complexity is linear, or at most quadratic.  

In addition, an Earley parser is ideal for grammars that may change
dynamically.  Earley parsing requires little or no preprocessing of
the grammar, unlike LALR(k) parsers that must generate parser states
at preprocessing time.

\subsection{Terminology}

The terminology we use is slightly different from that used by Earley.

An \emph{input token} or \emph{token} for short is an object that is
produced by the \emph{tokenizer} (also known as the \emph{scanner},
but we shall use ``tokenizer'' in order to avoid confusion with the
parser function called the \emph{scanner} by Earley), such as the
number $123$, the identifier \texttt{hello} or a left parenthesis.

A \emph{terminal symbol} or \emph{terminal} for short, is the
\emph{type} of an input token such as \emph{variable} or
\emph{number}.  How to organize tokens into types is sometimes a
matter of taste, and often a terminal has only one instance (such as
left parenthesis).  

A \emph{nonterminal symbol} or \emph{nonterminal} for short, is a
symbol that appears on the left-hand side of a grammar rule.

A \emph{grammar rule} or a \emph{production} is defined as in the
literature with a left-hand side in the form of a \emph{nonterminal}
and a right-hand side which is a (possibly empty) sequence of
terminals and nonterminals.  

An \emph{item} is similar to a grammar rule, except that there is a
dot either at the beginning, at the end, or between two symbols of the
right-hand side.

\subsection{How Earley parsing works}

An Earley parser is defined on $n+1$ sequential \emph{states} where
$n$ is the number of tokens of the input.  Each token determines the
transition from one state to the next one in the sequence.

For the purpose of this document, we shall name the states
sequentially from $0$, whereas in a real implementation, they would be
identified with their unique pointers. 

A \emph{suffix} is a pair consisting of an item and a state number.
The state number corresponds to the state in which the suffix was
first introduced. 

In the literature (and indeed by Earley himself), the transition from
one state to the next one is described as the sequential invocation of
three different parser functions, called respectively the
\emph{completer}, the \emph{predictor}, and the \emph{scanner}.  As we
shall see later on in this document, doing it this way requires a
special treatment for so-called $\epsilon$-rules, i.e., rules with an
empty sequence of symbols in the right-hand side.  We also give a
better method that does not have this problem.  In this section,
however, we stick to the traditional method.

The completer is invoked only on suffixes where the dot of the
corresponding item is at the end of the sequence of symbols.  We call
such an item \emph{completer item} and the corresponding suffix
\emph{completer suffix}.  

Its function is as follows.  When it sees a completer suffix of the
form $N \ra a_1 a_2 \ldots a_k \cdot, s$ where $N$ is a nonterminal,
$a_1 a_2 \ldots a_k$ is a sequence of symbols and $s$ is the state in
which the suffix $N \ra \cdot a_1 a_2 \ldots a_k, s$ was introduced by
prediction, it searches state $s$ for suffixes of the form 
$M \ra b_1 b_2 \ldots b_i \cdot N b_{i+2} \ldots b_l, t$.  For each
such suffix, it adds the suffix $M \ra b_1 b_2 \ldots b_i N \cdot
b_{i+2} \ldots b_l, t$ to the current state, unless it is already
present.  Here, suffixes are considered equal if they have the same
item and the same state of introduction.  

The predictor is invoked only on suffixes where the dot of the
corresponding item is followed by a nonterminal.  We call such an item
\emph{predictor item} and the corresponding suffix \emph{predictor
suffix}.

Its function is as follows.  When it sees a predictor suffix of the
form $M \ra b_1 b_2 \ldots b_i \cdot N b_{i+2} \ldots b_l, t$, it
looks in the grammar for rules of the form $N \ra a_1 a_2 \ldots a_k$,
and for each one, introduces the suffix $N \ra \cdot a_1 a_2 \ldots
a_k, s$, where $s$ is the number of the current state, unless such a
suffix is already present. 

The scanner is invoked only on suffixes where the dot of the
corresponding item is followed by a terminal.  We call such an item
\emph{scanner item} and the corresponding suffix \emph{scanner
suffix}.

Its function is as follows.  When it sees a scanner suffix of the form
$M \ra b_1 b_2 \ldots b_i \cdot T b_{i+2} \ldots b_l, t$, where $T$ is
the terminal between the current state and the state following the
current one, it adds the suffix $M \ra b_1 b_2 \ldots b_i T \cdot
b_{i+2} \ldots b_l, t$ to the state following the current one.

For the purpose of this section, we shall use a typical expression
grammar with operations for addition and multiplication:

\texttt{E $\ra$ T\\E $\ra$ E + T\\T $\ra$ F\\T $\ra$ T * F\\F $\ra$ V} 

where \texttt{E} stands for \emph{expression}, \texttt{T} for
\emph{term}, \texttt{F} for \emph{factor}, and \texttt{V} for
\emph{variable}. 

Parsing starts in state $0$ with an artificially created suffix with a
predictor suffix having some arbitrary left-hand side, and a right-hand
side that starts with the nonterminal we want the parser to recognize
and that ends with some end-of-file marker such as \texttt{\$}.  The
dot precedes the first symbol of the right-hand side.  Here, we are
going to use \texttt{\% $\rightarrow$$\cdot$E \$} as the artificial
suffix. 

Parsing stops in state $n$, which is detected by the fact that the
tokenizer returns end-of-file.  If in that state there exists a
scanner suffix (assuming the EOF marker is a terminal) of the form
\texttt{\% $\rightarrow$ E$\cdot$\$}, then parsing was successful.  If
not, parsing failed.  

Here, we represent parser states and tokens as rows of a table.  The
first row of table represents state $0$.  Thus, the initial state of
the parser looks like this:

\begin{tabular}{|l|l|l|l|}
\hline
token/state & Predictor suffixes & Completer suffixes & Scanner suffixes\\
\hline
0      & \texttt{\% $\rightarrow$$\cdot$E \$} & & \\
\hline
\texttt{V} &  & &\\
\hline
\end{tabular}

We always run first the completer, then the predictor, and finally the
scanner.  Here there are no completer suffixes, so we are done running
the completer.  Running the predictor, we get two new suffixes from
the rules with a left-hand side of \texttt{E}.

\begin{tabular}{|l|l|l|l|}
\hline
token/state & Predictor suffixes & Completer suffixes & Scanner suffixes\\
\hline
0     & \texttt{\% $\ra$$\cdot$E \$,0} & & \\
      & \texttt{E $\ra$$\cdot$T,0} & & \\
      & \texttt{E $\ra$$\cdot$E + T},0 & & \\
\hline
\texttt{V}     &  & &\\
\hline
\end{tabular}

One of the newly introduced suffixes again has an \texttt{E} after the
dot so no new suffixes are introduced from it.  Predicting from the
other one, we introduce suffixes with \texttt{T} on the left-hand
side. 

\begin{tabular}{|l|l|l|l|}
\hline
token/state & Predictor suffixes & Completer suffixes & Scanner suffixes\\
\hline
0     & \texttt{\% $\ra$$\cdot$E \$,0} & & \\
      & \texttt{E $\ra$$\cdot$T,0} & & \\
      & \texttt{E $\ra$$\cdot$E + T,0} & & \\
      & \texttt{T $\ra$$\cdot$F,0} & & \\
      & \texttt{T $\ra$$\cdot$T * F,0} & & \\
\hline
\texttt{V}     &  & &\\
\hline
\end{tabular}

Finally, predicting from the suffix with a \texttt{F} after the dot,
we introduce a scanner suffix (since the symbol after the dot is a terminal).

\begin{tabular}{|l|l|l|l|}
\hline
token/state & Predictor suffixes & Completer suffixes & Scanner suffixes\\
\hline
0     & \texttt{\% $\ra$$\cdot$E \$,0} & & \texttt{F $\ra$$\cdot$V,0}\\
      & \texttt{E $\ra$$\cdot$T,0} & & \\
      & \texttt{E $\ra$$\cdot$E + T,0} & & \\
      & \texttt{T $\ra$$\cdot$F,0} & & \\
      & \texttt{T $\ra$$\cdot$T * F,0} & & \\
\hline
\texttt{V}     &  & &\\
\hline
\end{tabular}

We are now ready to run the scanner.  The next input token is
\texttt{V}, so the scanner suffix with a \texttt{V} after the dot
generates a completer suffix in state 1. 

\begin{tabular}{|l|l|l|l|}
\hline
token/state & Predictor suffixes & Completer suffixes & Scanner suffixes\\
\hline
 0    & \texttt{\% $\ra$$\cdot$E \$,0} & & \texttt{F $\ra$$\cdot$V,0}\\
      & \texttt{E $\ra$$\cdot$T,0} & & \\
      & \texttt{E $\ra$$\cdot$E + T,0} & & \\
      & \texttt{T $\ra$$\cdot$F,0} & & \\
      & \texttt{T $\ra$$\cdot$T * F,0} & & \\
\hline
\texttt{V}     &  & &\\
\hline
1     &  &  \texttt{F $\ra$ V$\cdot$,0}& \\
\hline
\texttt{+} &  & &\\
\hline
\end{tabular}

We must now run the completer.  We should look for suffixes with an
\texttt{F} after the dot in state 0.  There is one, namely \texttt{T
$\ra$$\cdot$F,0} which generates the completer suffix \texttt{T $\ra$
F$\cdot$,0} in state 1.

\begin{tabular}{|l|l|l|l|}
\hline
token/state & Predictor suffixes & Completer suffixes & Scanner suffixes\\
\hline
0     & \texttt{\% $\ra$$\cdot$E \$,0} & & \texttt{F $\ra$$\cdot$V,0}\\
      & \texttt{E $\ra$$\cdot$T,0} & & \\
      & \texttt{E $\ra$$\cdot$E + T,0} & & \\
      & \texttt{T $\ra$$\cdot$F,0} & & \\
      & \texttt{T $\ra$$\cdot$T * F,0} & & \\
\hline
\texttt{V} &  & &\\
\hline
1     &  & \texttt{F $\ra$ V$\cdot$,0}& \\
      &  & \texttt{T $\ra$ F$\cdot$,0}& \\
\hline
\texttt{+} &  & &\\
\hline
\end{tabular}

Completing from the newly introduced suffix generates another completer
suffix, namely \texttt{E $\ra$ T$\cdot$,0} and a scanner suffix: 
\texttt{T $\ra$ T$\cdot$* F,0}.

\begin{tabular}{|l|l|l|l|}
\hline
token/state & Predictor suffixes & Completer suffixes & Scanner suffixes\\
\hline
0     & \texttt{\% $\ra$$\cdot$E \$,0} & & \texttt{F $\ra$$\cdot$V,0}\\
      & \texttt{E $\ra$$\cdot$T,0} & & \\
      & \texttt{E $\ra$$\cdot$E + T,0} & & \\
      & \texttt{T $\ra$$\cdot$F,0} & & \\
      & \texttt{T $\ra$$\cdot$T * F,0} & & \\
\hline
\texttt{V} &  & &\\
\hline
1     &  & \texttt{F $\ra$ V$\cdot$,0}& \texttt{T $\ra$ T$\cdot$* F,0}\\
      &  & \texttt{T $\ra$ F$\cdot$,0}& \\
      &  & \texttt{E $\ra$ T$\cdot$,0}& \\
\hline
\texttt{+} &  & &\\
\hline
\end{tabular}

Running the completer again gives us the scanner suffixes \texttt{\%
  $\ra$ E$\cdot$\$,0} and \texttt{E $\ra$ E$\cdot$+ T,0}

\begin{tabular}{|l|l|l|l|}
\hline
token/state & Predictor suffixes & Completer suffixes & Scanner suffixes\\
\hline
0     & \texttt{\% $\ra$$\cdot$E \$,0} & & \texttt{F $\ra$$\cdot$V,0}\\
      & \texttt{E $\ra$$\cdot$T,0} & & \\
      & \texttt{E $\ra$$\cdot$E + T,0} & & \\
      & \texttt{T $\ra$$\cdot$F,0} & & \\
      & \texttt{T $\ra$$\cdot$T * F,0} & & \\
\hline
\texttt{V} &  & &\\
\hline
1     &  & \texttt{F $\ra$ V$\cdot$,0}& \texttt{T $\ra$ T$\cdot$* F,0}\\
      &  & \texttt{T $\ra$ F$\cdot$,0}& \texttt{\% $\ra$ E$\cdot$\$,0}\\
      &  & \texttt{E $\ra$ T$\cdot$,0}& \texttt{E $\ra$ E$\cdot$+ T,0}\\
\hline
\texttt{+} &  & &\\
\hline
\end{tabular}

There are no predictor suffixes in state 1, so we run the scanner.  The
input symbol is \texttt{+} so only the suffix \texttt{E $\ra$
  E$\cdot$+ T,0} works, and generates the predictor suffix \texttt{E $\ra$ E
  +$\cdot$T,0} in state 2. 

\begin{tabular}{|l|l|l|l|}
\hline
token/state & Predictor suffixes & Completer suffixes & Scanner suffixes\\
\hline
0     & \texttt{\% $\ra$$\cdot$E \$,0} & & \texttt{F $\ra$$\cdot$V,0}\\
      & \texttt{E $\ra$$\cdot$T,0} & & \\
      & \texttt{E $\ra$$\cdot$E + T,0} & & \\
      & \texttt{T $\ra$$\cdot$F,0} & & \\
      & \texttt{T $\ra$$\cdot$T * F,0} & & \\
\hline
\texttt{V} &  & &\\
\hline
1     &  & \texttt{F $\ra$ V$\cdot$,0}& \texttt{T $\ra$ T$\cdot$* F,0} \\
      &  & \texttt{T $\ra$ F$\cdot$,0}& \texttt{E $\ra$ E$\cdot$+ T,0} \\
      &  & \texttt{E $\ra$ T$\cdot$,0}& \\
\hline
\texttt{+} &  & &\\
\hline
 2    & \texttt{E $\ra$ E +$\cdot$T,0} & & \\
\hline 
\texttt{V} &  & &\\
\hline
\end{tabular}

There are no completer suffixes in state 2, so we go directly to
running the predictor.  Predicting from the nonterminal \texttt{T}, we
get two more predictor suffixes, namely \texttt{T $\ra$$\cdot$F,2} and
\texttt{T $\ra$$\cdot$T * F,2} as well as the scanner suffix \texttt{F
  $\ra$$\cdot$V,2}. 

\begin{tabular}{|l|l|l|l|}
\hline
token/state & Predictor suffixes & Completer suffixes & Scanner suffixes\\
\hline
0     & \texttt{\% $\ra$$\cdot$E \$,0} & & \texttt{F $\ra$$\cdot$V,0}\\
      & \texttt{E $\ra$$\cdot$T,0} & & \\
      & \texttt{E $\ra$$\cdot$E + T,0} & & \\
      & \texttt{T $\ra$$\cdot$F,0} & & \\
      & \texttt{T $\ra$$\cdot$T * F,0} & & \\
\hline
\texttt{V} &  & &\\
\hline
1     &  & \texttt{F $\ra$ V$\cdot$,0}& \texttt{T $\ra$ T$\cdot$* F,0} \\
      &  & \texttt{T $\ra$ F$\cdot$,0}& \texttt{E $\ra$ E$\cdot$+ T,0} \\
      &  & \texttt{E $\ra$ T$\cdot$,0}& \\
\hline
\texttt{+} &  & &\\
\hline
2     & \texttt{E $\ra$ E +$\cdot$T} & & \texttt{F $\ra$$\cdot$V,2}\\
      & \texttt{T $\ra$$\cdot$F,2} & & \\
      & \texttt{T $\ra$$\cdot$T * F,2} & & \\
\hline 
\texttt{V} &  & &\\
\hline
\end{tabular}


We now run the scanner.  The input token is a \texttt{V}, so we get
the following situation:

\begin{tabular}{|l|l|l|l|}
\hline
token/state & Predictor suffixes & Completer suffixes & Scanner suffixes\\
\hline
0     & \texttt{\% $\ra$$\cdot$E \$,0} & & \texttt{F $\ra$$\cdot$V,0}\\
      & \texttt{E $\ra$$\cdot$T,0} & & \\
      & \texttt{E $\ra$$\cdot$E + T,0} & & \\
      & \texttt{T $\ra$$\cdot$F,0} & & \\
      & \texttt{T $\ra$$\cdot$T * F,0} & & \\
\hline
\texttt{V} &  & &\\
\hline
1     &  & \texttt{F $\ra$ V$\cdot$,0}& \texttt{T $\ra$ T$\cdot$* F,0} \\
      &  & \texttt{T $\ra$ F$\cdot$,0}& \texttt{E $\ra$ E$\cdot$+ T,0} \\
      &  & \texttt{E $\ra$ T$\cdot$,0}& \\
\hline
\texttt{+} &  & &\\
\hline
2     & \texttt{E $\ra$ E +$\cdot$T,0} & & \texttt{F $\ra$$\cdot$V,2}\\
      & \texttt{T $\ra$$\cdot$F,2} & & \\
      & \texttt{T $\ra$$\cdot$T * F,2} & & \\
\hline 
\texttt{V} & & &\\
\hline
3     &  &  \texttt{F $\ra$ V$\cdot$,2}& \\
\hline
\texttt{*} & & & \\
\hline
\end{tabular}

Running the completer, we must look for predictor suffixes in state $2$
with a \texttt{F} after the dot.  We find \texttt{T $\ra$$\cdot$F,2},
which generates the new completer suffix \texttt{T $\ra$ F$\cdot$,2}.
Running it again on this new suffix generates suffixes \texttt{E $\ra$
E + T$\cdot$,0} (completer) and \texttt{T $\ra$ T$\cdot$* F,2}
(scanner).  The new completer suffix says we must go back to state $0$
and look for predictor suffixes with an \texttt{E} following the dot.
We find \texttt{\% $\ra$$\cdot$E \$,0}, which will generate the
scanner suffix \texttt{\% $\ra$ E$\cdot$\$,0} and \texttt{E
  $\ra$$\cdot$E + T,0} which will generate the scanner suffix 
\texttt{E $\ra$ E$\cdot$+ T,0}

\begin{tabular}{|l|l|l|l|}
\hline
token/state & Predictor suffixes & Completer suffixes & Scanner suffixes\\
\hline
0     & \texttt{\% $\ra$$\cdot$E \$,0} & & \texttt{F $\ra$$\cdot$V,0}\\
      & \texttt{E $\ra$$\cdot$T,0} & & \\
      & \texttt{E $\ra$$\cdot$E + T,0} & & \\
      & \texttt{T $\ra$$\cdot$F,0} & & \\
      & \texttt{T $\ra$$\cdot$T * F,0} & & \\
\hline
\texttt{V} &  & &\\
\hline
1     &  & \texttt{F $\ra$ V$\cdot$,0}& \texttt{T $\ra$ T$\cdot$* F,0} \\
      &  & \texttt{T $\ra$ F$\cdot$,0}& \texttt{E $\ra$ E$\cdot$+ T,0} \\
      &  & \texttt{E $\ra$ T$\cdot$,0}& \\
\hline
\texttt{+} &  & &\\
\hline
2     & \texttt{E $\ra$ E +$\cdot$T,0} & & \texttt{F $\ra$$\cdot$V,2}\\
      & \texttt{T $\ra$$\cdot$F,2} & & \\
      & \texttt{T $\ra$$\cdot$T * F,2} & & \\
\hline 
\texttt{V} & & &\\
\hline
3     &  &  \texttt{F $\ra$ V$\cdot$,2}& \texttt{T $\ra$ T$\cdot$* F,2}\\
      &  &  \texttt{T $\ra$ F$\cdot$,2}& \texttt{\% $\ra$ E$\cdot$\$,0}\\
      &  &  \texttt{E $\ra$ E + T$\cdot$,0}& \texttt{E $\ra$ E$\cdot$+ T,0}\\
\hline
\texttt{*} & & & \\
\hline
\end{tabular}

There are no predictor suffixes, so we run the scanner.  The input is
\texttt{*}, so only the scanner suffix \texttt{T $\ra$ T$\cdot$* F,2}
works.  This gives the following situation:

\begin{tabular}{|l|l|l|l|}
\hline
token/state & Predictor suffixes & Completer suffixes & Scanner suffixes\\
\hline
0     & \texttt{\% $\ra$$\cdot$E \$,0} & & \texttt{F $\ra$$\cdot$V,0}\\
      & \texttt{E $\ra$$\cdot$T,0} & & \\
      & \texttt{E $\ra$$\cdot$E + T,0} & & \\
      & \texttt{T $\ra$$\cdot$F,0} & & \\
      & \texttt{T $\ra$$\cdot$T * F,0} & & \\
\hline
\texttt{V} &  & &\\
\hline
1     &  & \texttt{F $\ra$ V$\cdot$,0}& \texttt{T $\ra$ T$\cdot$* F,0} \\
      &  & \texttt{T $\ra$ F$\cdot$,0}& \texttt{E $\ra$ E$\cdot$+ T,0} \\
      &  & \texttt{E $\ra$ T$\cdot$,0}& \\
\hline
\texttt{+} &  & &\\
\hline
2     & \texttt{E $\ra$ E +$\cdot$T,0} & & \texttt{F $\ra$$\cdot$V,2}\\
      & \texttt{T $\ra$$\cdot$F,2} & & \\
      & \texttt{T $\ra$$\cdot$T * F,2} & & \\
\hline 
\texttt{V} & & &\\
\hline
3     &  &  \texttt{F $\ra$ V$\cdot$,2}& \texttt{T $\ra$ T$\cdot$* F,2}\\
      &  &  \texttt{T $\ra$ F$\cdot$,2}& \texttt{\% $\ra$ E$\cdot$\$,0}\\
      &  &  \texttt{E $\ra$ E + T$\cdot$,0}& \texttt{E $\ra$ E$\cdot$+ T,0}\\
\hline
\texttt{*} & & & \\
\hline
4     & \texttt{T $\ra$ T *$\cdot$F,2} & & \\
\hline
\texttt{V} &  & &\\
\hline
\end{tabular}

There are no completer suffixes, so we run the predictor.  We get:

\begin{tabular}{|l|l|l|l|}
\hline
token/state & Predictor suffixes & Completer suffixes & Scanner suffixes\\
\hline
0     & \texttt{\% $\ra$$\cdot$E \$,0} & & \texttt{F $\ra$$\cdot$V,0}\\
      & \texttt{E $\ra$$\cdot$T,0} & & \\
      & \texttt{E $\ra$$\cdot$E + T,0} & & \\
      & \texttt{T $\ra$$\cdot$F,0} & & \\
      & \texttt{T $\ra$$\cdot$T * F,0} & & \\
\hline
\texttt{V} &  & &\\
\hline
1     &  & \texttt{F $\ra$ V$\cdot$,0}& \texttt{T $\ra$ T$\cdot$* F,0} \\
      &  & \texttt{T $\ra$ F$\cdot$,0}& \texttt{E $\ra$ E$\cdot$+ T,0} \\
      &  & \texttt{E $\ra$ T$\cdot$,0}& \\
\hline
\texttt{+} &  & &\\
\hline
2     & \texttt{E $\ra$ E +$\cdot$T,0} & & \texttt{F $\ra$$\cdot$V,2}\\
      & \texttt{T $\ra$$\cdot$F,2} & & \\
      & \texttt{T $\ra$$\cdot$T * F,2} & & \\
\hline 
\texttt{V} & & &\\
\hline
3     &  &  \texttt{F $\ra$ V$\cdot$,2}& \texttt{T $\ra$ T$\cdot$* F,2}\\
      &  &  \texttt{T $\ra$ F$\cdot$,2}& \texttt{\% $\ra$ E$\cdot$\$,0}\\
      &  &  \texttt{E $\ra$ E + T$\cdot$,0}& \texttt{E $\ra$ E$\cdot$+ T,0}\\
\hline
\texttt{*} & & & \\
\hline
4     & \texttt{T $\ra$ T *$\cdot$F,2} & & \texttt{F $\ra$$\cdot$V,4}\\
\hline
\texttt{V} &  & &\\
\hline
\end{tabular}

Now we run the scanner.  The input it \texttt{V}, so we get the
following situation:

\begin{tabular}{|l|l|l|l|}
\hline
token/state & Predictor suffixes & Completer suffixes & Scanner suffixes\\
\hline
0     & \texttt{\% $\ra$$\cdot$E \$,0} & & \texttt{F $\ra$$\cdot$V,0}\\
      & \texttt{E $\ra$$\cdot$T,0} & & \\
      & \texttt{E $\ra$$\cdot$E + T,0} & & \\
      & \texttt{T $\ra$$\cdot$F,0} & & \\
      & \texttt{T $\ra$$\cdot$T * F,0} & & \\
\hline
\texttt{V} &  & &\\
\hline
1     &  & \texttt{F $\ra$ V$\cdot$,0}& \texttt{T $\ra$ T$\cdot$* F,0} \\
      &  & \texttt{T $\ra$ F$\cdot$,0}& \texttt{E $\ra$ E$\cdot$+ T,0} \\
      &  & \texttt{E $\ra$ T$\cdot$,0}& \\
\hline
\texttt{+} &  & &\\
\hline
2     & \texttt{E $\ra$ E +$\cdot$T,0} & & \texttt{F $\ra$$\cdot$V,2}\\
      & \texttt{T $\ra$$\cdot$F,2} & & \\
      & \texttt{T $\ra$$\cdot$T * F,2} & & \\
\hline 
\texttt{V} & & &\\
\hline
3     &  &  \texttt{F $\ra$ V$\cdot$,2}& \texttt{T $\ra$ T$\cdot$* F,2}\\
      &  &  \texttt{T $\ra$ F$\cdot$,2}& \texttt{\% $\ra$ E$\cdot$\$,0}\\
      &  &  \texttt{E $\ra$ E + T$\cdot$,0}& \texttt{E $\ra$ E$\cdot$+ T,0}\\
\hline
\texttt{*} & & & \\
\hline
4     & \texttt{T $\ra$ T *$\cdot$F,2} & & \texttt{F $\ra$$\cdot$V,4}\\
\hline
\texttt{V} & & &\\
\hline
5     & & \texttt{F $\ra$ V$\cdot$,4} &\\
\hline
\texttt{\$} &  & &\\
\hline
\end{tabular}

We must now run the completer which generates the new completer suffix
\texttt{T $\ra$ T * F$\cdot$,2}.  Going to state $2$ to look for a
\texttt{T}, we generate \texttt{E $\ra$ E + T$\cdot$,0} and \texttt{T
  $\ra$ T$\cdot$* F,2}.  Going to state $0$ to look for \texttt{E} we
generate \texttt{\% $\ra$ E$\cdot$\$,0} and \texttt{E $\ra$ E$\cdot$+
  T,0}.  We now have the following situation:


\begin{tabular}{|l|l|l|l|}
\hline
token/state & Predictor suffixes & Completer suffixes & Scanner suffixes\\
\hline
0     & \texttt{\% $\ra$$\cdot$E \$,0} & & \texttt{F $\ra$$\cdot$V,0}\\
      & \texttt{E $\ra$$\cdot$T,0} & & \\
      & \texttt{E $\ra$$\cdot$E + T,0} & & \\
      & \texttt{T $\ra$$\cdot$F,0} & & \\
      & \texttt{T $\ra$$\cdot$T * F,0} & & \\
\hline
\texttt{V} &  & &\\
\hline
1     &  & \texttt{F $\ra$ V$\cdot$,0}& \texttt{T $\ra$ T$\cdot$* F,0} \\
      &  & \texttt{T $\ra$ F$\cdot$,0}& \texttt{E $\ra$ E$\cdot$+ T,0} \\
      &  & \texttt{E $\ra$ T$\cdot$,0}& \\
\hline
\texttt{+} &  & &\\
\hline
2     & \texttt{E $\ra$ E +$\cdot$T,0} & & \texttt{F $\ra$$\cdot$V,2}\\
      & \texttt{T $\ra$$\cdot$F,2} & & \\
      & \texttt{T $\ra$$\cdot$T * F,2} & & \\
\hline 
\texttt{V} & & &\\
\hline
3     &  &  \texttt{F $\ra$ V$\cdot$,2}& \texttt{T $\ra$ T$\cdot$* F,2}\\
      &  &  \texttt{T $\ra$ F$\cdot$,2}& \texttt{\% $\ra$ E$\cdot$\$,0}\\
      &  &  \texttt{E $\ra$ E + T$\cdot$,0}& \texttt{E $\ra$ E$\cdot$+ T,0}\\
\hline
\texttt{*} & & & \\
\hline
4     & \texttt{T $\ra$ T *$\cdot$F,2} & & \texttt{F $\ra$$\cdot$V,4}\\
\hline
\texttt{V} & & &\\
\hline
5     & & \texttt{F $\ra$ V$\cdot$,4} & \texttt{T $\ra$ T$\cdot$* F,2}\\
      & & \texttt{T $\ra$ T * F$\cdot$,2} & \texttt{\% $\ra$ E$\cdot$\$,0}\\
      & & \texttt{E $\ra$ E + T$\cdot$,0} & \texttt{E $\ra$ E$\cdot$+ T,0}\\
\hline
\texttt{\$} &  & &\\
\hline
\end{tabular}

We are now at end-of-file of the input.  The suffix \texttt{\% $\ra$
  E$\cdot$\$,0} is present, so the parse was successful. 

\subsection{Dealing with $\epsilon$-rules}

\subsection{Ambiguous grammars}

Consider the grammar:

\texttt{E $\ra$ A B\\A $\ra$ x\\A $\ra$ x y\\B $\ra$ z\\B $\ra$ y z} 

This grammar is ambiguous, because the input \texttt{x y z} can be
generated either as \texttt{E $\ra$ A B $\ra$ x B $\ra$ x y z} or as 
\texttt{E $\ra$ A B $\ra$ x y B $\ra$ x y z}.  

Let us see what happens when we give the Earley parser an ambiguous
grammar like that. 

\begin{tabular}{|l|l|l|l|}
\hline
token/state & Predictor suffixes & Completer suffixes & Scanner suffixes\\
\hline
0      & \texttt{\% $\rightarrow$$\cdot$E\$ \$} & & \\
\hline
\texttt{x} &  & &\\
\hline
\end{tabular}

\begin{tabular}{|l|l|l|l|}
\hline
token/state & Predictor suffixes & Completer suffixes & Scanner suffixes\\
\hline
0 & \texttt{\% $\rightarrow$$\cdot$E \$,0 \$} & & \texttt{A $\ra$$\cdot$x,0} \\
  & \texttt{E $\ra$$\cdot$A B,0} & & \texttt{A $\ra$$\cdot$x y,0}\\
\hline
\texttt{x} &  & &\\
\hline
1 & \texttt{E $\ra$ A$\cdot$B,0}& \texttt{A $\ra$ x$\cdot$,0} & \texttt{A $\ra$ x$\cdot$y,0}\\
  & & & \texttt{B $\ra$$\cdot$z, 1}\\
  & & & \texttt{B $\ra$$\cdot$y z, 1}\\
\hline
\texttt{y} &  & &\\
\hline
\end{tabular}

\begin{tabular}{|l|l|l|l|}
\hline
token/state & Predictor suffixes & Completer suffixes & Scanner suffixes\\
\hline
0 & \texttt{\% $\rightarrow$$\cdot$E \$,0 \$} & & \texttt{A $\ra$$\cdot$x,0} \\
  & \texttt{E $\ra$$\cdot$A B,0} & & \texttt{A $\ra$$\cdot$x y,0}\\
\hline
\texttt{x} &  & &\\
\hline
1 & \texttt{E $\ra$ A$\cdot$B,0}& \texttt{A $\ra$ x$\cdot$,0} & \texttt{A $\ra$ x$\cdot$y,0}\\
  & & & \texttt{B $\ra$$\cdot$z, 1}\\
  & & & \texttt{B $\ra$$\cdot$y z, 1}\\
\hline
\texttt{y} &  & &\\
\hline
2 & \texttt{E $\ra$ A$\cdot$B,0} & \texttt{A $\ra$ x y$\cdot$,0} & \texttt{B $\ra$ y$\cdot$z, 1}\\
  &  & & \texttt{B $\ra$$\cdot$z, 2}\\
  &                              & & \texttt{B $\ra$$\cdot$y z, 2}\\
\hline
\texttt{z} &  & &\\
\hline
\end{tabular}

\begin{tabular}{|l|l|l|l|}
\hline
token/state & Predictor suffixes & Completer suffixes & Scanner suffixes\\
\hline
0 & \texttt{\% $\rightarrow$$\cdot$E \$,0 \$} & & \texttt{A $\ra$$\cdot$x,0} \\
  & \texttt{E $\ra$$\cdot$A B,0} & & \texttt{A $\ra$$\cdot$x y,0}\\
\hline
\texttt{x} &  & &\\
\hline
1 & \texttt{E $\ra$ A$\cdot$B,0}& \texttt{A $\ra$ x$\cdot$,0} & \texttt{A $\ra$ x$\cdot$y,0}\\
  & & & \texttt{B $\ra$$\cdot$z, 1}\\
  & & & \texttt{B $\ra$$\cdot$y z, 1}\\
\hline
\texttt{y} &  & &\\
\hline
2 & \texttt{E $\ra$ A$\cdot$B,0} & \texttt{A $\ra$ x y$\cdot$,0} & \texttt{B $\ra$ y$\cdot$z, 1}\\
  &  & & \texttt{B $\ra$$\cdot$z, 2}\\
  &                              & & \texttt{B $\ra$$\cdot$y z, 2}\\
\hline
\texttt{z} &  & &\\
\hline
3 & & \texttt{B $\ra$ y z$\cdot$, 1} & \\
  & & \texttt{B $\ra$ z$\cdot$, 2} & \\
  & & \texttt{E $\ra$ A B$\cdot$,0} &\\
  & & \texttt{\% $\rightarrow$ E$\cdot$\$,0} &\\
\hline
\end{tabular}

The parse is successful.  Notice how the suffix \texttt{E $\ra$ A
  B$\cdot$,0} in state $3$ was generated by the completer in two
  different ways, from \texttt{B $\ra$ y z$\cdot$, 1} and \texttt{B
  $\ra$ z$\cdot$, 2} respectively.  This is the general pattern for
  ambiguities.  Whenever a suffix in two or more different ways by the
  completer there is an ambiguity.  

Ambiguities are bad because they give rise to multiple parse trees for
a single grammar on a single input sequence.  In fact, there can be
exponentially many parse trees compared to the size of the grammar.
To see that, consider the grammar:

\texttt{E $\ra$ N N N N N N N N\\N $\ra$ A\\N $\ra$ B\\A $\ra$ x\\B $\ra$ x} 

We can avoid the ambiguity problem by not trying to build \emph{all}
the parse trees, but only \emph{one}.  

\subsection{Building the parse tree}

\section{Protocols}

\Defclass {grammar}

\Defclass {rule}

\Defgeneric {rules} {grammar}

Given a grammar, returns a list of the rules of the
grammar. {\ReadOnly}

\Defgeneric {target} {rule}

Given a rule, returns the target symbol.  

\bibliography{earley}{}
\bibliographystyle{alpha}

\printindex

\end{document}
